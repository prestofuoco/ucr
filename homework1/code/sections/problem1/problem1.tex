\section{Developing RAW images}
In this problem, we will use the provided RAW image \texttt{Thayer.CR2} 
in order to implement a basic image processing pipeline.

\subsection{Implementing a basic image processing pipeline}


\textbf{RAW image conversion.}
Calling \texttt{dcraw -4 -d -v -w -T Thayer.CR2} 
converts the RAW image to a TIFF file without any color interpolation. 
We observe the CLI output:
\begin{Verbatim}[fontsize=\small]
    Loading Canon EOS 2000D image from Thayer.CR2 ...
    Scaling with darkness 2044, saturation 16383, and
    multipliers 2.165039 1.000000 1.643555 1.000000
    Building histograms...
    Writing data to Thayer.tiff ...
\end{Verbatim}
where our multipliers represent \texttt{<r\_scale> <g\_scale> <b\_scale> <g\_scale>},
and darkness and saturation represent the black and white levels, respectively.

We then call \texttt{dcraw -4 -D -T Thayer.CR2} to convert the RAW image to a TIFF file 
without any color interpolation or white balancing, obtaining a grayscale image that  
we will use for the remainder of the problem.\\


\noindent\textbf{Python initials.}
Using \texttt{skimage}'s \texttt{imread}, we are able to read the image and obtain its values. 
We can then apply a linear transformation to the image so that the value \texttt{<black>} is 
mapped to 0, and the value \texttt{<white>} is mapped to 1. We then clip the negative 
values to 0, and values greater than 1 to 1.

We can achieve this by recalling our values from the reconnaissance run of \texttt{dcraw} and
writing the transformation as:
\begin{equation}
    \text{linearized image} = \frac{\text{image} - \texttt{<black>}}{\texttt{<white>} - \texttt{<black>}}
\end{equation}
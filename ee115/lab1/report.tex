\documentclass[11pt]{article}
\usepackage[utf8]{inputenc}
\usepackage{hyperref, amsmath, amssymb, amsthm, graphicx, fancyvrb, enumitem, titlesec, setspace, float, fancyvrb, minted}
\usepackage[dvipsnames]{xcolor}
\usepackage[top=1in, bottom=1in, left=1.25in, right=1.25in]{geometry}

\titleformat{\section}{\normalfont\bfseries}{}{0em}{}
\titlespacing*{\section}{0pt}{1.5ex plus .2ex minus .2ex}{0.8ex plus .1ex}
\begin{document}
\noindent Andre Winkel \hfill \today \\
\rule{\textwidth}{0.4pt}

\begin{center} \large {\textbf{Lab 1}} \\[0em] {EE115, Introduction to Communication Systems, Fall 2025} \end{center}

\section{Problem 1: Examining a random signal and AM power efficiency}
In this problem, we will examine the average power of a random signal that has
its minimum value larger than or equal to $-1$, and also examine its impact on the power
efficiency of conventional amplitude modulation (AM) signals.
\begin{enumerate}[label=\textbf{\alph*)}, leftmargin=2.6em]
    \item We begin by using the Gaussian-random-number generator to generate a random sequence
    \begin{equation}
        m[1], m[2], \dots, m[N]
    \end{equation}
    where we will set $N$ as $200$ [a large integer].
    We can do so in MATLAB by employing the \texttt{}

    \item
\end{enumerate}


\end{document}